
\documentclass[a4paper, 12pt]{article}
\usepackage{a4wide}
\usepackage {amsmath}
\usepackage{amssymb}
\usepackage {graphicx}
\usepackage[utf8]{inputenc} 
\usepackage[francais]{babel}
\usepackage{fancyhdr}
\usepackage{setspace}
\usepackage{fancyhdr}
\usepackage{lastpage}
\usepackage{extramarks}
\usepackage{chngpage}
\usepackage{soul}
\usepackage{algorithmicx} 
\usepackage{algpseudocode} 
\usepackage{multicol}
\usepackage[usenames,dvipsnames]{color}
\usepackage{graphicx,float,wrapfig}
\usepackage{ifthen}
\usepackage{listings}
\usepackage{courier}
\usepackage{esint}
\usepackage{bbm}
\usepackage{graphics}
\usepackage{graphicx}
\usepackage{subfig}
\usepackage{epsfig}
\usepackage{pgf,tikz}
\usetikzlibrary{arrows}
\usepackage{braket}
\usepackage{MnSymbol,wasysym}
\usepackage{marvosym}
\usepackage{dsfont}
\usepackage{stmaryrd}

\lhead{} 
\chead{} 
\rhead{\bfseries Fourier et convolution} 
\lfoot{JC Toussaint - Phelma}
%\cfoot{} 
%\rfoot{\thepage}

% This is the color used for MATLAB comments below
\definecolor{MyDarkGreen}{rgb}{0.0,0.4,0.0}

% For faster processing, load Matlab syntax for listings
\lstloadlanguages{Matlab}%
\lstset{language=Matlab,                        % Use MATLAB
        frame=single,                           % Single frame around code
        basicstyle=\small\ttfamily,             % Use small true type font
        keywordstyle=[1]\color{Blue}\bf,        % MATLAB functions bold and blue
        keywordstyle=[2]\color{Purple},         % MATLAB function arguments purple
        keywordstyle=[3]\color{Blue}\underbar,  % User functions underlined and blue
        identifierstyle=,                       % Nothing special about identifiers
                                                % Comments small dark green courier
        commentstyle=\usefont{T1}{pcr}{m}{sl}\color{MyDarkGreen}\small,
        stringstyle=\color{Purple},             % Strings are purple
        showstringspaces=false,                 % Don't put marks in string spaces
        tabsize=5,                              % 5 spaces per tab
        %
        %%% Put standard MATLAB functions not included in the default
        %%% language here
        morekeywords={xlim,ylim,var,alpha,factorial,poissrnd,normpdf,normcdf},
        %
        %%% Put MATLAB function parameters here
        morekeywords=[2]{on, off, interp},
        %
        %%% Put user defined functions here
        morekeywords=[3]{FindESS, homework_example},
        %
        morecomment=[l][\color{Blue}]{...},     % Line continuation (...) like blue comment
        numbers=left,                           % Line numbers on left
        firstnumber=1,                          % Line numbers start with line 1
        numberstyle=\tiny\color{Blue},          % Line numbers are blue
        stepnumber=5                            % Line numbers go in steps of 5
        }

% Includes a MATLAB script.
% The first parameter is the label, which also is the name of the script
%   without the .m.
% The second parameter is the optional caption.
\newcommand{\matlabscript}[2]
  {\begin{itemize}\item[]\lstinputlisting[caption=#2,label=#1]{#1.m}\end{itemize}}

\pagestyle{fancy}

\begin{document}

\bibliographystyle{alpha}

\title{Transformée de fourier rapide et convolution}

\author{Jean-Christophe Toussaint
%  Phelma\\
%\texttt{jean-christophe.toussaint@phelma.grenoble-inp.fr}
}
%\date{\today}
%\date{\vspace{-10ex}}
 
\maketitle

La technique usuelle pour déterminer les fréquences caractéristiques ainsi que l’amplitude des harmoniques qui composent un signal, consiste à calculer sa transformée de fourier. Dans une approche numérique, cette opération est effectuée sur un ensemble de valeurs échantillonnées.

Le but de l'exercice est dans un premier temps, d'utiliser ici le module scipy.fft pour calculer les transformées de fourier rapides directe et inverse dont la complexité est en $\vartheta(N \ln N)$,
puis d'effectuer des produits de convolution en utilisant le module scipy.signal.



\section{Définitions mathématiques}

La transformée directe d'un signal périodique $s$ échantillonné en N points 
est définie par :

\begin{equation} 
ft[k]=\sum_{n=0}^{N-1}  s[n] \exp \left(-i \frac{2\pi k}{N}n \right)
\end{equation} 
où $k \in  \llbracket  0, N-1 \rrbracket $ avec $i^2=-1$.

Sa transformée inverse est définie par :
\begin{equation} 
s[n]=\frac{1}{N} \sum_{k=0}^{N-1}  ft[n] \exp \left(+i \frac{2\pi k}{N}n \right)
\end{equation}

C'est en 1965 que  James Cooley et John Tukey redécouvrent une méthode de factorisation de ces sommes
discrètes, déjà imaginée par Gauss en 1805, qui permettent de réduire à $\vartheta(N \ln N)$, le nombre d'opérations.
Cette technique porte le nom de transformée de fourier rapide (Fast Fourier Transforms).

Le produit de convolution de deux signaux échantillonnés en N points,   est définie par:
\begin{equation} 
(f*g)[p]=\sum_{m=0}^{n-1} f_m  . g_{p-m}
\label{convolution}
\end{equation}

où $p-m \equiv p-m+n \mod n$
avec $p \in [0, n[$.

On montre aussi que la transformée de fourier d'un produit de convolution est égale au produit des
transformées de fourier.

\section{Exercice : transformée de Fourier d'un sinus}

On veut calculer la transformée de fourier discrète d'un signal sinusoïdal périodique $G(t)~=~sin(\frac{2\pi}{T} t)$. On fixera $t \in [0, T[$.

Il est nécessaire d'importer les modules numpy, matplotlib et scipy.fft :

\begin{lstlisting}[language=Python]
%matplotlib inline
import numpy as np
import matplotlib.pyplot as plt
from scipy.fft import fft, fftfreq
\end{lstlisting}

\begin{enumerate} 
\item Echantillonner le domaine $ [0, T=1[$ en $N=100$ points en utilisant la fonction numpy
{\tt np.linspace}. Utiliser la commande help(cmd) pour obtenir une aide en ligne sur tout objet ou
fonction ou module.
\item Utiliser les fonctions {\tt fft.fft} et {\tt fft.ifft} pour calculer 
les transformées directe et inverse.
\item Tracer le signal et les composantes réelle et imaginaire de sa transformée de fourier.
\item Vérifier que l'on reconstruit bien le signal initial $G(t)$ en appliquant la transformée inverse 
sur le spectre de fourier.
\end{enumerate} 

\section{Exercice : transformée de Fourier d'une gaussienne}

Faites de même sur un signal gaussien $G(t)=\exp(\frac{-t^2}{2 \sigma^2})$.
On prendra $\sigma=0.1$ et on fixera $t \in [-T/2, T/2[$.

\begin{enumerate} 
\item Echantillonner le domaine $ [-T/2, T/2[$ en $N=100$ points en utilisant la fonction numpy
{\tt np.linspace}. 
\item Avant d'appliquer la transformée de fourier, il est nécessaire de décaler vers la gauche
le tableau contenant $G$ de $N/2$ cases en utilisant la fonction {\tt np.roll}.
Une explication vous êtes donnée en annexe I.

%Son origine est liée à la définition de la transformée de fourier fenêtrée:
%\begin{equation}
%\hat{f}(\omega)=\int_{-T/2}^{+T/2} s(t) \exp \left(-i \omega t \right) dt
%\end{equation}
%
%avec $\omega_k=\frac{2 \pi}{T} k$ avec  $k \in  \llbracket  0, N-1 \rrbracket $.
%
%La discrétisation en temps impose que $t=n T/N -T/2$ avec $n \in  \llbracket  0, N-1 \rrbracket $.
%On obtient :
%\begin{equation}
%\hat{f}(k)= \frac{T}{N} \exp \left(+i \omega_k (T/2) \right)  \sum_{n=0}^{N-1} s[n] \exp \left(-i \frac{2 \pi k}{N} n \right) =
%\frac{T}{N} \exp \left(+i \omega_k (T/2) \right) fft[k]
%\end{equation}


%$$
%fft(k)= \frac{T}{N} \sum_{n=0}^{N-1} s(n T/N -T/2) \exp \left(-i \omega_k (n T/N -T/2) \right) 
%$$
%
%$$
%\hat{f}(k)= \frac{T}{N} \exp \left(+i \omega (T/2) \right)  \sum_{n=0}^{N-1} s[n] \exp \left(-i \omega_k (n T/N) \right) 
%$$
%
%$$
%\hat{f}(k)= \frac{T}{N} \exp \left(+i \omega_k (T/2) \right)  \sum_{n=0}^{N-1} s[n] \exp \left(-i \frac{2 \pi k}{N} n \right) 
%$$

\item Utiliser les fonctions {\tt fft.fft} et {\tt fft.ifft} pour calculer 
les transformées directe et inverse.
\item Tracer le signal et les composantes réelle et imaginaire de sa transformée de fourier.
\end{enumerate} 

\section{Convolution de deux signaux}
 
 On définit ci-après un signal carré comme une liste de 1 complétée à droite d'une même quantité de zéros.
On utilise une méthode de zero-padding pour doubler la taille du tableau tout en le complétant de zéros. 
Cette technique a peu d'intérêt en 1D mais devient très utile pour les dimensions supérieures à 1.
 
\begin{lstlisting}[language=Python]
sq=np.ones(5)
sq=np.pad(sq, (0, len(sq)), mode='constant') # (before, after)
\end{lstlisting}
 
 \begin{enumerate} 
\item Développer une fonction python {\tt conv(f, g)} permettant de faire le calcul du produit de convolution discret d'après
\eqref{convolution}.

\item La fonction {\tt  numpy.convolve(sq, sq, mode='full')} effectue une convolution discrète au sens mathématique.
Comparer avec le résultat précédent.

\item La fonction {\tt  scipy.ndimage.convolve1d} effectue le calcul du produit de convolution au sens du traitement d'image, sans retournement du noyau, ce qui ne correspond pas à la définition mathématique. Il faut donc retourner le noyau qui est passé en deuxième argument de {\tt convolve1d(sq, sq[::-1], mode='constant', cval=0) }.

\item La fonction {\tt scipy.signal.fftconvolve(sq, sq, mode='full')} du module \break
{\tt scipy.signal} permet d'obtenir directement le produit de convolution. Le zero-padding sur {\tt sq} est appliqué en interne. 

\item Vérifier que l'on obtient aussi la même résultat en utilisant la propriété sur la transformée de fourier d'un produit de convolution.
 \end{enumerate} 
 
\section{Convolution d'un signal par une gaussienne}

Dans le cadre d'un modèle simpliste unidimensionnel, on imagine ici qu'une grandeur $\phi$ (nommée ci-après potentiel) est obtenue par 
le produit de convolution suivant :
\begin{equation}
\phi(x)=\int_{-L}^{L} q(y) G(x-y) dy = (q*G) (x)
\end{equation}
où $q(x)$ est l'équivalent d'une distribution de charges et $G(x)$ est appelé le noyau.
Pour fixer les idées, on prendra dans la suite un noyau gaussien $G(x)=\exp(\frac{-x^2}{2 \sigma^2})$, défini partout.

Lorsque la distribution de charges  est la somme de distributions de dirac, $q(x) = \sum_{i} q_i \delta(x-x_i)$, 
l'expression du potentiel $\phi$ s'écrit comme la somme de gaussiennes centrées en $x_i$:

\begin{equation}
\phi(x)=\sum_i q_i \exp \left(\frac{-(x-x_i)^2}{2 \sigma^2} \right) 
\end{equation}

Le but de l'exercice est de reproduire ce résultat, en plaçant les charges $q_i$ aux noeuds $x_i$ d'une grille unidimensionnelle.
 
\begin{enumerate} 
\item Discrétisée $[-L, L=1]$ en $N=100$ points d'espace en utilisant la fonction {\tt linspace}
\item Définir le tableau numpy $q$ de dimension $N$ dans lequel on placera quelques charges non nulles.
\item Défnir le tableau numpy $G$ de dimension $N$ contenant l'échantillonnage de la fonction de Green. On prendra $\sigma=0.1$.
\item Utiliser d'abord votre propre fonction {\tt conv} calculant le produit de convolution sans optimisation.
\item Utiliser ensuite {\tt filter.convolve1d(q, G, mode='constant')}. Comparer vos résultats sous la forme de graphiques.
\end{enumerate} 


On désire maintenant développer notre propre routine de convolution, utilisant les FFT du module {\tt fft}.
Montrer qu'il est nécessaire 
\begin{enumerate} 
 \item de mettre en oeuvre la technique du zéro-padding aux tableaux $q$ et $G$. 
\item d'appliquer sur le tableau $G$ un décalage à gauche de $N/2$ termes. Tracer
$G$ et commenter.
\item de calculer les FFT directes sur $q$ et $G$.
\item d'appliquer une FFT inverse sur leur produit. 
\item et finalement de ne garder que les N premiers éléments du tableau précédent, pour obtenir le produit de convolution désiré.
\end{enumerate} 

\section{Annexe I : décalage vers la gauche}

Son origine est liée à la définition de la transformée de fourier fenêtrée:
\begin{equation}
\hat{f}(\omega)=\int_{-T/2}^{+T/2} s(t) \exp \left(-i \omega t \right) dt
\end{equation}

avec $\omega_k=\frac{2 \pi}{T} k$ avec  $k \in  \llbracket  0, N-1 \rrbracket $.

La discrétisation en temps impose que $t=n T/N -T/2$ avec $n \in  \llbracket  0, N-1 \rrbracket $.
Le pas de temps $dt$ est donc égal à $T/N$.

On obtient :
\begin{equation}
\hat{f}(\omega_k)= \frac{T}{N} \exp \left(+i \omega_k (T/2) \right)  \sum_{n=0}^{N-1} s[n] \exp \left(-i \frac{2 \pi k}{N} n \right) =
\frac{T}{N} \exp \left(+i \omega_k (T/2) \right) ft [k]
\end{equation}

% 
% 
%\begin{equation}
%\hat{f}(\omega)=\int_{-T/2}^{+T/2} s(t) \exp \left(-i \omega t \right) dt
%\end{equation}
       
% \matlabscript{FDhelmM1etud}{Embryon de programme}
\end{document}

