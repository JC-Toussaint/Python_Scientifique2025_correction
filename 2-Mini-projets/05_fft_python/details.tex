
\documentclass[a4paper, 12pt]{article}
\usepackage{a4wide}
\usepackage {amsmath}
\usepackage{amssymb}
\usepackage {graphicx}
\usepackage[utf8]{inputenc} 
\usepackage[francais]{babel}
\usepackage{fancyhdr}
\usepackage{setspace}
\usepackage{fancyhdr}
\usepackage{lastpage}
\usepackage{extramarks}
\usepackage{chngpage}
\usepackage{soul}
\usepackage{algorithmicx} 
\usepackage{algpseudocode} 
\usepackage{multicol}
\usepackage[usenames,dvipsnames]{color}
\usepackage{graphicx,float,wrapfig}
\usepackage{ifthen}
\usepackage{listings}
\usepackage{courier}
\usepackage{esint}
\usepackage{bbm}
\usepackage{graphics}
\usepackage{graphicx}
\usepackage{subfig}
\usepackage{epsfig}
\usepackage{pgf,tikz}
\usetikzlibrary{arrows}
\usepackage{braket}
\usepackage{MnSymbol,wasysym}
\usepackage{marvosym}
\usepackage{dsfont}
\usepackage{stmaryrd}

\lhead{} 
\chead{} 
\rhead{\bfseries Fourier et convolution} 
\lfoot{JC Toussaint - Phelma}
%\cfoot{} 
%\rfoot{\thepage}

% This is the color used for MATLAB comments below
\definecolor{MyDarkGreen}{rgb}{0.0,0.4,0.0}

% For faster processing, load Matlab syntax for listings
\lstloadlanguages{Matlab}%
\lstset{language=Matlab,                        % Use MATLAB
        frame=single,                           % Single frame around code
        basicstyle=\small\ttfamily,             % Use small true type font
        keywordstyle=[1]\color{Blue}\bf,        % MATLAB functions bold and blue
        keywordstyle=[2]\color{Purple},         % MATLAB function arguments purple
        keywordstyle=[3]\color{Blue}\underbar,  % User functions underlined and blue
        identifierstyle=,                       % Nothing special about identifiers
                                                % Comments small dark green courier
        commentstyle=\usefont{T1}{pcr}{m}{sl}\color{MyDarkGreen}\small,
        stringstyle=\color{Purple},             % Strings are purple
        showstringspaces=false,                 % Don't put marks in string spaces
        tabsize=5,                              % 5 spaces per tab
        %
        %%% Put standard MATLAB functions not included in the default
        %%% language here
        morekeywords={xlim,ylim,var,alpha,factorial,poissrnd,normpdf,normcdf},
        %
        %%% Put MATLAB function parameters here
        morekeywords=[2]{on, off, interp},
        %
        %%% Put user defined functions here
        morekeywords=[3]{FindESS, homework_example},
        %
        morecomment=[l][\color{Blue}]{...},     % Line continuation (...) like blue comment
        numbers=left,                           % Line numbers on left
        firstnumber=1,                          % Line numbers start with line 1
        numberstyle=\tiny\color{Blue},          % Line numbers are blue
        stepnumber=5                            % Line numbers go in steps of 5
        }

% Includes a MATLAB script.
% The first parameter is the label, which also is the name of the script
%   without the .m.
% The second parameter is the optional caption.
\newcommand{\matlabscript}[2]
  {\begin{itemize}\item[]\lstinputlisting[caption=#2,label=#1]{#1.m}\end{itemize}}

\pagestyle{fancy}

\begin{document}

\bibliographystyle{alpha}

\title{Transformée de fourier rapide et convolution}

\author{Jean-Christophe Toussaint
%  Phelma\\
%\texttt{jean-christophe.toussaint@phelma.grenoble-inp.fr}
}
%\date{\today}
%\date{\vspace{-10ex}}
 
\maketitle

\section{Annexe I : décalage vers la gauche}

Son origine est liée à la définition de la transformée de fourier fenêtrée:
\begin{equation}
\hat{f}(\omega)=\int_{-T/2}^{+T/2} s(t) \exp \left(-i \omega t \right) dt
\end{equation}

avec $\omega_k=\frac{2 \pi}{T} k$ avec  $k \in  \llbracket  0, N-1 \rrbracket $.

La discrétisation en temps impose que $t=n T/N -T/2$ avec $n \in  \llbracket  0, N-1 \rrbracket $.
Le pas de temps $dt$ est donc égal à $T/N$.

On obtient :
\begin{equation}
\hat{f}(\omega_k)= \frac{T}{N} \exp \left(+i \omega_k (T/2) \right)  \sum_{n=0}^{N-1} s[n] \exp \left(-i \frac{2 \pi k}{N} n \right) =
\frac{T}{N} \exp \left(+i \omega_k (T/2) \right) ft [k]
\end{equation}

\section{Annexe II : produit de convolution}

Son origine est liée à la définition de la transformée de fourier fenêtrée:
\begin{equation}
\widehat{f*g}(\omega)=\lim_{L \rightarrow \infty} \int_{-L}^{+L} 
dy\ \int_{-L}^{+L}  dx\ 
f(y) g(x-y) \exp \left(-i \omega x \right) 
\end{equation}

avec $\omega_k=\frac{2 \pi}{T} k$ avec  $k \in  \llbracket  0, N-1 \rrbracket $.

\begin{equation}
\widehat{f*g}(\omega)=\lim_{L \rightarrow \infty} \int_{-L}^{+L} 
dy\  f(y)
\int_{-L}^{+L} dx\ 
g(x-y) \exp \left(-i \omega x \right) 
\end{equation}

\begin{equation}
\widehat{f*g}(\omega)=\lim_{L \rightarrow \infty} \int_{-L}^{+L} 
dy\  f(y)
\int_{-L-y}^{+L-y} dx'\
g(x') \exp \left(-i \omega (x'+y) \right) 
\end{equation}

$x'=x-y$

\begin{equation}
\widehat{f*g}(\omega)=\lim_{L \rightarrow \infty} \int_{-L}^{+L} 
dy\  f(y) \exp \left(-i \omega y \right) 
\int_{-L-y}^{+L-y} dx'\
g(x') \exp \left(-i \omega x'\right) 
\end{equation}

$x'=x-y$

Dans l'hypothèse où $g(x)$ a été périodisée, on obtient:
\begin{equation}
\widehat{f*g}(\omega)=\hat{f}(\omega) \ \hat{g}(\omega)
\end{equation}

%La discrétisation en temps impose que $t=n T/N -T/2$ avec $n \in  \llbracket  0, N-1 \rrbracket $.
%Le pas de temps $dt$ est donc égal à $T/N$.
%
%On obtient :
%\begin{equation}
%\hat{f}(\omega_k)= \frac{T}{N} \exp \left(+i \omega_k (T/2) \right)  \sum_{n=0}^{N-1} s[n] \exp \left(-i \frac{2 \pi k}{N} n \right) =
%\frac{T}{N} \exp \left(+i \omega_k (T/2) \right) ft [k]
%\end{equation}

% 
% 
%\begin{equation}
%\hat{f}(\omega)=\int_{-T/2}^{+T/2} s(t) \exp \left(-i \omega t \right) dt
%\end{equation}
       
% \matlabscript{FDhelmM1etud}{Embryon de programme}
\end{document}

