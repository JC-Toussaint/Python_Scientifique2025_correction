
\documentclass[a4paper, 10pt]{article}
\usepackage {amsmath}
\usepackage{amsfonts}
\usepackage{esint}
\usepackage{a4wide}
\usepackage {amsmath}
\usepackage{amssymb}
\usepackage {graphicx}
\usepackage[utf8]{inputenc} 
\usepackage[francais]{babel}
\usepackage{fancyhdr}
\usepackage{setspace}
\usepackage{fancyhdr}
\usepackage{lastpage}
\usepackage{extramarks}
\usepackage{chngpage}
\usepackage{soul}
\usepackage{algorithmicx} 
\usepackage{algpseudocode} 
\usepackage{multicol}
\usepackage[usenames,dvipsnames]{color}
\usepackage{graphicx,float,wrapfig}
\usepackage{ifthen}
\usepackage{listings}
\usepackage{courier}
\usepackage{esint}
\usepackage{bbm}
\usepackage{graphics}
\usepackage{graphicx}
\usepackage{subfig}
\usepackage{epsfig}
\usepackage{pgf,tikz}
\usetikzlibrary{arrows}
\usepackage{braket}
\usepackage{MnSymbol,wasysym}
\usepackage{marvosym}
\usepackage{dsfont}
\usepackage{listings}
\usepackage{color}


\definecolor{MyGreen}{rgb}{0.,0.35,0.}

    %%configuration de listings
    \lstset{
    language=c++,
   xleftmargin=-20pt, 
   xrightmargin=-20pt, 
   frame=single,                           % Single frame around code
    basicstyle=\ttfamily\small, %
    identifierstyle=\color{black}, %
    keywordstyle=\color{blue}, %
    stringstyle=\color{black!60}, %
    commentstyle=\color{MyGreen!100}, %
    columns=fixed, %
    tabsize=8, %
    extendedchars=true, %
    showspaces=false, %
    showstringspaces=false, %
    numbers=left, %
    numberstyle=\tiny, %
    breaklines=true, %
    breakautoindent=true, %
    captionpos=t
    }

% The first parameter is the label, which also is the name of the script
% The second parameter is the optional caption.
\newcommand{\insertlisting}[2]
  {\begin{itemize}\item[]\lstinputlisting[caption=#2,label=#1]{#1}\end{itemize}}

\newcommand{\norm}[1]{\lVert#1\rVert}

\lhead{} 
\chead{} 
\rhead{\bfseries Algèbre Linéaire en Python} 
\lfoot{J.-C. Toussaint - Phelma}
%\cfoot{} 
%\rfoot{\thepage}

\pagestyle{fancy}

\begin{document}


\title{Algèbre Linéaire en Python}

% \author{Quentin Rafhay - Jean-Christophe Toussaint\\
%  Phelma\\
%\texttt{jean-christophe.toussaint@phelma.grenoble-inp.fr}
%}
\date{\today}
 
\maketitle


\section{Résolution de systèmes d'équations linéaires}

\subsection{Méthode de Gauss-Jordan}

La méthode de Gauss-Jordan est une technique de résolution de systèmes d'équations linéaires:

\begin{equation}
\left(
\begin{array}{c c c c}
a_{0,0}   & a_{0,1}   & \cdots & a_{0,n-1} \\
a_{1,0}   & a_{1,1}   & \cdots & a_{1,n-1} \\
\vdots         & \vdots         & \ddots & \vdots \\
a_{n-1,0} & a_{n-1,1} & \cdots & a_{n-1,n-1} 
\end{array}
\right)
\left(
\begin{array}{c}
x_0 \\ x_1 \\ \vdots \\ x_{n-1}
\end{array}
\right)
=
\left(
\begin{array}{c}
b_0 \\ b_1 \\ \vdots \\ b_{n-1}
\end{array}
\right)
\end{equation}


Elle consiste à transformer la matrice associée $A=(a_{ij})_{\; 0\leq i < n,\;  0\leq j < n }$ en une matrice unité en effectuant des combinaisons linéaires de lignes.
Cette variante de la méthode de Gauss permet de calculer l'inverse d'une matrice.

\subsection{Matrice augmentée}

Soient $A$ et $B$ deux matrices  ayant le même nombre de lignes, on appelle matrice augmentée la matrice $(A | B)$ formée des deux blocs $A$ et $B$ mis côte à côte.

Cette notion est très pratique lorsqu'on veut résoudre le système d'équations avec différents seconds
membres. La matrice $B$ est alors construite en accollant à droite de $A$, les vecteurs colonnes formant les seconds membres.
Elle se réduit naturellement à un vecteur dans le cas d'un seul second membre.

\subsection{Algorithme de base sans changement de pivot (niveau *)}

On augmente la matrice $A$ avec le(s) vecteur(s) colonne(s) du second membre.
On vérifie que $\forall i,  a_{ii} \neq 0$ sinon effectue des échanges de lignes. 

L'algorithme est itératif et s'effectue en $n$ étapes. A l'étape $k \in [0, n[$, on
combine toutes les lignes sauf la ligne k pour faire apparaître des $0$ sur toute
la colonne $k$ sauf au niveau du pivot $a_{kk}$.

\begin{tabbing}
\hspace{1cm}\= \hspace{1cm}\= \kill
pour k=0 à n-1 \\
\> diviser la ligne $k$ de la matrice $A$ par $a_{kk}$ \\
\> pour i=0 à n-1 sauf k \\
\> \> retrancher à la ligne i, la nouvelle ligne k multipliée par $a_{ik}$ \\
\> fin pour i \\
fin pour k \\
\end{tabbing}

Après résolution, la matrice $A$ apparait comme la concaténation de la matrice
identité, accolée aux solutions $x$ du système.

\subsection{Exemple pratique}

Soit à résoudre:
\[
\left \{
\begin{array}{l c r}
2x+y-4z & = & 8 \\
3x+3y-5z & = & 14 \\
4x+5y-2z & = & 16 \\
\end{array}
\right .
\]

La matrice augmentée $A^{(0)}$ s'écrit :
\begin{equation}
A^{(0)}=
\left(
\begin{array}{c c c c }
2  & 1   & -4  \\
3  & 3   & -5   \\
4  & 5  & -2  \\ 
\end{array}
\right|
\left.
\begin{array}{c}
8  \\ 14   \\ 16  \\ 
\end{array}
\right)
\end{equation}

Divisons la première ligne par le pivot $a_{0,0}$ :
$l_0 \leftarrow {1 \over 2} \times l_0$

\begin{equation}
A^{(1)}=
\left(
\begin{array}{c c c c }
1  & {1/2}   & -2  \\
0  &  {3/2}   & 1   \\
0  & 3  & 6  \\ 
\end{array}
\right|
\left.
\begin{array}{c}
4   \\ 2    \\ 0  \\ 
\end{array}
\right)
\begin{array}{c}
   \\ l_1 \leftarrow l_1 -3 \times l_0   \\ l_2 \leftarrow l_2 -4 \times l_0  \\ 
\end{array}
\end{equation}

Divisons la deuxième ligne par le pivot $a_{11}$ :
$l_1 \leftarrow {2 \over 3} \times l_1$

\begin{equation}
A^{(2)}=
\left(
\begin{array}{c c c c }
1  & 0   & -7/3  \\
0  & 1   & 2/3   \\
0  & 0  &  4  \\ 
\end{array}
\right|
\left.
\begin{array}{c}
{10/3}   \\ {4/3}    \\ -4  \\ 
\end{array}
\right)
\begin{array}{c}
l_0 \leftarrow  l_0 -{1 \over 2}  \times l_1   \\     \\ l_2 \leftarrow  l_2 -3 \times l_1  \\ 
\end{array}
\end{equation}

Divisons la troisième ligne par le pivot $a_{2,2}$ : $l_2 \leftarrow {1 \over 4} \times l_2$

\begin{equation}
A^{(3)}=
\left(
\begin{array}{c c c c }
1  & 0   & 0  \\
0  & 1   & 0   \\
0  & 0  &  1  \\ 
\end{array}
\right|
\left.
\begin{array}{c}
1   \\ 2    \\ -1  \\ 
\end{array}
\right)
\begin{array}{c}
l_0 \leftarrow  l_0 +{7 \over 3}  \times l_2   \\   l_1 \leftarrow  l_1 - {2 \over 3} \times l_2   \\  \\ 
\end{array}
\end{equation}

La solution se trouve dans la quatrième colonne. Vérifier que $x=\left(1, 2, -1 \right)^t$ est bien solution.

\subsection{Inverse d'une matrice (niveau **)}

Le calcul direct de l'inverse d'une matrice $A$ mettant en oeuvre la formule
$A^{-1}= {1 \over det(A)} ^tco(A)$ est peu utilisé en pratique. 
On préfère résoudre $n$ systèmes linéaires en même temps en plaçant dans le second membre
la matrice identité $I_n$. On applique la méthode de Gauss-Jordan à la matrice augmentée $(A| I_n)$.
\\

 $\bullet$ Montrer que l'inverse de la matrice 

$A=\left(
\begin{array}{c c c }
2  & 1   & -4  \\
3  & 3   & -5   \\
4  & 5  &  -2  \\ 
\end{array}
\right)$
est 
$A^{-1}= {1 \over 12} \left(
\begin{array}{c c c }
19  & -18   & 7  \\
-14  & 12   & -2   \\
3  & -6  &  3  \\ 
\end{array}
\right)$.
Vérifier que $A A^{-1}=I$.

\subsection{Implémentation de Gauss-Jordan en Python (niveau *)}

\begin{enumerate} 
 \item Proposez une implémentation d'une fonction {\tt gauss\_jordan} en python
 de l'algorithme précédent, permettant de résoudre un système d'équations linéaires.
 On limitera la précision d'affichage avec l'instruction :
{\tt numpy.set\_printoptions(precision=4, suppress=True)}.
L'option {\tt suppress=True} supprime les petits zéros lors de l'affichage.
 \item Traitez quelques cas tests,  vérifier vos résultats obtenus avec la fonction {\tt numpy.linalg.solve}  
%\insertlisting{gauss_jordan.m}{Gauss Jordan}

\item On étudie ici la stabilité numérique de l'algorithme en prenant :\\
$A=\left(
\begin{array}{c c c c}
1  & 1/4   &   0 & 0  \\
1  & 5/4   & 12 & 0   \\
1  & 1/3  &  1   & 1  \\
1 & 5/4  & 13  &  1  
\end{array}
\right)$
et $B=\left(
\begin{array}{c c c c}
0  \\
0    \\
1  \\
0  
\end{array}
\right)$

Calculer le résidu $B-A\; x$ dans ce cas, commenter.
 
\item On émettra une erreur si le pivot $|a_{k,k}|< \epsilon \approx 10^{-6}$.
Pour ce faire on utilisera l'instruction {\tt raise} dans la fonction{\tt gauss\_jordan} 
pour lancer une erreur de type {\tt ZeroDivisionError}.
\begin{lstlisting}[language=Python]
 if abs(akk)<epsilon:
    raise ZeroDivisionError("pivot akk trop faible")
\end{lstlisting}

On récupérera l'erreur de type {\tt ZeroDivisionError} 
dans le programme appelant en utilisant la structure:
\begin{lstlisting}[language=Python]
try:
    gauss_jordan(K, L)
except ZeroDivisionError as err:
    print('Handling run-time error:', err)
\end{lstlisting}

\end{enumerate}
 
\subsection{Algorithme de Gauss-Jordan avec changement de pivot (niveau **)}

On augmente la matrice $A$ avec le(s) vecteur(s) colonne(s) du second membre.
On propose une nouvelle version de l'algorithme de Gauss-Jordan
avec maximisation du pivot.
L'algorithme est itératif et s'effectue en $n$ étapes. A l'étape $k \in [0, n[$, on
combine toutes les lignes sauf la ligne k pour faire apparaître des $0$ sur toute
la colonne $k$ sauf au niveau du pivot $a_{kk}$.

\begin{tabbing}
\hspace{1cm}\= \hspace{1cm}\= \kill
pour k=0 à n-1 \\
\> identifier l'indice de ligne $i \in [k, n[$  correspondant au $\max \{|a_{i,k}|\}$  \\
\> échanger les lignes $i$ et $k$ de la matrice $A$ \\
\> diviser ligne $k$ de la nouvelle matrice $A$ par $a_{kk}$ \\
\> pour i=0 à n-1 sauf k \\
\> \> retrancher à la ligne i, la nouvelle ligne k multipliée par $a_{ik}$ \\
\> fin pour i \\
fin pour k \\
\end{tabbing}
 
 \begin{enumerate} 
 \item traiter quelques cas tests.
 \item refaites l'étude de la stabilité numérique.
 \end{enumerate}
 
 \subsection{Résolution de systèmes mal conditionnés (niveau **)}
 
 On examine ici l'influence d'une perturbation $\delta b$ sur les termes du second membre $b$
 du  système d'équations linéaires. Il en résulte une variation $\delta x$ sur la solution.
 On montre par contre que $\delta x$ n'est pas forcément une perturbation de $x$ et
 dépend du conditionnement de $A$.
 On a l'inégalité suivante : 
 \begin{equation}
 \frac{\norm {\delta x}}{\norm x} < cond(A) \frac{\norm {\delta b}}{\norm b}
 \label{conderr}
 \end{equation}.
 
Le conditionnement est donné dans le cas d'une matrice symétrique
 par le rapport des valeurs propres  extrèmes en norme de $A$:
 $\kappa(A) = \frac{|\lambda_{max}(A)|}{|\lambda_{min}(A)|}$.
 
  \begin{enumerate} 
 \item Résoudre le système d'équations en prenant :
 $A=\left(
\begin{array}{c c c c}
10  & 7   &   8 & 7  \\
7  &  5   & 6 & 5   \\
8  &  6  &  10   & 9  \\
7 & 5  & 9  &  10  
\end{array}
\right)$
et $B=\left(
\begin{array}{c c c c}
32  \\
23   \\
33  \\
31 
\end{array}
\right)$ 
 \item Faites une fluctuation de $\pm 1\%$ sur le second membre et en déduire l'erreur
 sur la solution. 
 \item Calculer le conditionnement de $A$ à partir de son spectre. On utilisera la
 fonction \\ {\tt numpy.linalg.eig} pour calculer les valeurs propres.
 \item Pour vérifier l'inégalité \eqref{conderr},  on calculera les normes avec la fonction
 {\tt  numpy.linalg.norm} en choisissant une norme infinie, par exemple.
 \end{enumerate}
 
\section{Décomposition LU d'une matrice avec pivot (niveau ***)}

Le but est ici de décomposer une matrice sous la forme : $A=P\; L\; U$
où $P$ est une matrice de permutation de lignes de la matrice unité,
$L$ une matrice triangulaire inférieure ne contenant que des $1$ sur sa diagonale et 
$U$ une matrice triangulaire supérieure.
La matrice de permutation est une matrice orthogonale et vérifie la propriété
$P^{-1}=P^t$.

A titre d'exemple, factorisons la matrice
 $A=\left(
\begin{array}{c c c}
10  & -7   &  0  \\
-3  &  2   & 6   \\
 5  &  -1  &  5  
\end{array}
\right)$

Déterminons le pivot qui correspond à l'amplitude maximale des termes
 de la première sous-colonne. La première étape de permutation est triviale 
 parce que la valeur $10$ est déjà la plus grande valeur. la première
 matrice de permutation est l'identité. 
 
\begin{equation}
 P_0\; A^{(0)}=
 \left(
\begin{array}{c c c}
1  & 0   &  0  \\
0  &  1   & 0  \\
0  &  0  &  1  
\end{array}
\right)
 \left(
\begin{array}{c c c}
10  & -7   &  0  \\
-3  &  2   & 6   \\
 5  &  -1  &  5  
\end{array}
\right)
=
 \left(
\begin{array}{c c c}
10  & -7   &  0  \\
-3  &  2   & 6   \\
 5  &  -1  &  5  
\end{array}
\right)
\end{equation}

La première étape d'élimination consiste à faire les combinaisons linéaires suivantes
-- Numérotation des lignes conformes à python --: 
 \begin{equation}
 \begin{array}{c}
 l_0 \leftarrow l_0 \\
   l_1 \leftarrow l_1 +3/10 \times l_0   \\ l_2 \leftarrow l_2 -1/2 \times l_0  \\ 
\end{array}
\end{equation}

\begin{equation}
 A^{(1)}=L^{-1}_0 \;  P_0 \; A^{(0)}=
  \left(
\begin{array}{c c c}
1  & 0   &  0  \\
3/10  &  1   & 0  \\
-1/2  &  0  &  1  
\end{array}
\right)
 \left(
\begin{array}{c c c}
10  & -7   &  0  \\
-3  &  2   & 6   \\
 5  &  -1  &  5  
\end{array}
\right)
=
 \left(
\begin{array}{c c c}
10  & -7   &  0  \\
0  &  -1/10   & 6   \\
0  &  5/2  &  5  
\end{array}
\right)
\end{equation}

La seconde étape de permutation revient à échanger les lignes $1$ et $2$:

\begin{equation}
P_1\; A^{(1)}=
 \left(
\begin{array}{c c c}
1  & 0   &  0  \\
0  &  0   & 1  \\
0  &  1  &  0  
\end{array}
\right)
 \left(
\begin{array}{c c c}
10  & -7   &  0  \\
0  &  -1/10   & 6   \\
0  &  5/2  &  5  
\end{array}
\right)
=
 \left(
\begin{array}{c c c}
10  & -7   &  0  \\
0  &  5/2  &  5  \\
0  &  -1/10   & 6   
\end{array}
\right)
\end{equation}

La seconde étape d'élimination consiste à faire l'opération suivante: 
\begin{equation}
 \begin{array}{c}
  l_2 \leftarrow l_2 +\frac{1}{25} \times l_1 
\end{array}
\end{equation}

\begin{equation}
 U=A^{(2)}=L^{-1}_1 \;  P_1 \; A^{(1)}=
  \left(
\begin{array}{c c c}
1  & 0   &  0  \\
0  &  1   & 0  \\
0  &  1/25  &  1  
\end{array}
\right)
 \left(
\begin{array}{c c c}
10  & -7   &  0  \\
0  &  5/2  &  5  \\
0  &  -1/10   & 6   
\end{array}
\right)
=
 \left(
\begin{array}{c c c}
10  & -7   &  0  \\
0  &  5/2   & 5   \\
0  &  0  &  31/5  
\end{array}
\right)
\end{equation}

\noindent On construit finalement la matrice globale de permutation $P=P_0\; P_1$. \\
Le calcul des matrices $L_k$ s'effectue facilement grâce à la propriété suivante:
 \begin{equation}
 L_k=L_k^{-1}+2I_n
 \end{equation}
 La matrice triangulaire inférieure $L$ est finalement reconstruite en appliquant la relation suivante :
  \begin{equation}
 L= P^t\;  \Pi_{k=0}^{n-2}  (P_k \; L_k)
 \end{equation}

 
Dans l'algorithme fourni ci-après, on définit une fonction {\tt Find\_Pivot(A, k)}
qui retourne pour une colonne $k$ donné, l'indice de ligne $i \in [k, n[$  
correspondant au $\max \{|a_{i,k}|\}$.

\begin{algorithmic}[1]
   \begin{multicols}{2}
\Function{LU}{$A$}
    \State $PL \gets I_n $
    \State $P \gets I_n $
    
    \For{$k \in [0, n-1[$}
        \State $P_k \gets I_n $
        \State $i \gets \Call{Find\_Pivot}{$$A, k$$} $
        \State $P_k \gets \Call{Swap\_Lines}{$$P_k,i, k$$} $
        \State $A \gets \Call{Swap\_Lines}{$$A,i, k$$} $ \\
        \State $invL_k \gets I_n$ 
            \For{$i \in [k+1, n-1[$}
                    \State $invL_k(i,k) \gets -a_{i,k}/a_{k,k} $
            \EndFor \\
        \State $A \gets \Call{Mult}{$$invL_k$, $A$$} $   
        \State $L_k \gets -invL_k + 2 I_n $
        \State $PL_k \gets  \Call{Mult}{$$P_k, L_k$$} $
        \State $PL \gets \Call{Mult}{$$PL, PL_k$$} $    
        \State $P \gets \Call{Mult}{$$P, P_k$$} $ 
                 
      \EndFor\\

      \State $U \gets A $
      \State $invP \gets \Call{Transpose}{$P$} $
      \State $L \gets \Call{Mult}{$$invP, PL$$} $ 
                                
    \State \Return $P, L, U$

\EndFunction
\end{multicols}
\end{algorithmic}

  \begin{enumerate} 
 \item Proposez une implémentation en python.
 \item Testez-la en utilisant les matrices fournies précédemment.
  \end{enumerate} 
 
\end{document}

